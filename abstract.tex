The X-ray survey technique is designed to systematically survey the position
of photodetectors installed as part of the calorimeter upgrade of the MEG experiment.
The photosensitive surface is scanned in axial (Z) and azimuthal ($\phi$) directions
with a precisely controlled, thin collimated beam of X-ray obtained from \cob X-ray source.
Systematic shifts as well as small deformations are observed 
which result from the assembly of photodetectors and the cryogenic cooling of
of the liquid Xenon scintillator. Analysis of data shows in-situ positions of the photodetectors
deviate from the expectation by 2~mm in Z coordinate and a maximum 5~mm in $\phi$
coordinate. An uncertainty of 0.3~mm is attributed to the calculated positions
well within the resolution goals for photon reconstruction in second run of the MEG experiment.

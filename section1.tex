\section{\label{intro}Introduction}
The MEG experiment is designed to search for charged flavor violating
\mueg process (cLFV) in muon decays \cite{megexperiment}.  While
exceedingly small branching ratio ($B < 10^{-50}$) postulated by the
Standard Model (SM) prevents the charged flavor exchange process to be
observed directly, it provides a clean background free mode of
searching for beyond the Standard Model physics.  Extensions to the
Standard Model with Supersymmetry (SUSY) have been posited to explain
the apparent lightness of neutrino masses and flavor violation via
oscillations, as well as for the broader goal of unifying all the SM
gauge groups. Searches for new physics with cLFV processes are
underway motivated by the fact that the rates of such processes
enhanced by SUSY are just within the design sensitivities of the
current experiments.  Results from the LHC so far have ruled out the
existence of supersymmetric particles in the mass range 1-2 TeV
\cite{lhcsusy1,lhcsusy2}, and failed to observe charged flavor violating
process\cite{lhcclfv1,lhcclfv2}.  Given the upper limit of
direct detection at the LHC, indirect searches such as MEG provide a
powerful and complimentary probe of investigating cLFV.  To-date no
evidence has been observed, the world’s most sensitive limit was set
by the MEG experiment $B < 4.2 \times 10^{-14}$ at 90\% CL using data
collected in the running period 2009-2013 (Run~I) \cite{meg1result}.
An upgraded detector components and beam conditions in the MEG
experiment aim to improve this measurement by an order of magnitude in
the second run (2019-2022).  The entrance face of the liquid Xe (LXe)
calorimeter of MEG was upgraded with MPPC (Multi-Pixel Photon counter)
photodetectors to provide improved photon detection and
reconstruction.  This paper describes the X-ray survey conducted to
determine absolute in-situ positions of these photodetectors.

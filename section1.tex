\section{\label{intro}Introduction}
The MEG \cite{megexperiment} and MEG II\cite{meg2experiment} experiments search for the decay of an anti-muon to a positron and a photon ( \mueg ). This is an example of a charged lepton flavor violating
process (cLFV) in which additive quantum numbers associated with each type of lepton (e, $\mu$, or $\tau$) are not conserved in processes involving charged leptons. We already know that lepton flavor is violated for neutrinos, as evidenced by the observation of neutrino oscillations. The observation of neutrino oscillations implies a very small \mueg branching fraction due to neutrino mixing ( ($B(\mueg) ~ 10^{-504}$, much too small to be detected. Hence, this decay provides a clean, background free mode for
searching for new, non Standard Model physics.  Extensions to the
Standard Model, for example supersymmetry (SUSY) have been posited largely for the purpose of unifying all the SM
gauge groups, and also allow significantly larger rates for cLFV processes. Searches for processes that violate muon number have been done for over 50 years since the realization that the muon is essentially a heavy electron; these searches are now further motivated by the fact that the rates of such processes could be close to current limits in many models for new physics. Results from the LHC so far have ruled out the
existence of super-symmetric particles in the mass range 1-2 TeV
\cite{lhcsusy1,lhcsusy2}, and failed to observe charged flavor violating
process\cite{lhcclfv1,lhcclfv2}. Indirect searches such as MEG provide a
powerful and complimentary probe of investigating cLFV at levels much lower than possible at the LHC.  To-date no
evidence of cLFV processes has been observed; the world’s most sensitive limit was set
by the MEG experiment $B < 4.2 \times 10^{-13}$ CHECK THIS at 90\% CL \cite{meg1result}.

Upgraded detector components and beam conditions in the MEG II
experiment aim to improve this measurement by an order of magnitude in
the next few years. An essential part of the improved detector is better precision in the measurement of kinematic properties of the positron and photon, essential to eliminating events that might fake the signal.  
In this paper, we will describe a novel technique for better determining the precise position of the photo-detectors in the MEG liquid xenon (LXe) calorimeter that is used to measure the time, position, and energy of the photon. MEG II will use an improved calorimeter that includes replacing photo-detectors on the entrance face of the calorimeter with MPPCs (Multi-Pixel Photon Counters) to provide improved photon detection and position and time measurement. 
reconstruction.  

In the following sections we introduce the calorimeter used to measure the photon kinematics and motivate the requirement for improved knowledge of the photo-detectors, describe the novel technique for measuring their position, describe the technical implementation of the measuring apparatus, describe the data collected, describe the analysis and results of the position measurement, report on some useful ancillary measurements that we found we could do with x-ray alignment apparatus, and summarize the results.  
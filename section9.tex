\section{\label{conclusion}Conclusion}
A novel approach for position survey of photodetectors and its
application to the LXe calorimeter of the MEG experiment is described.
It is performed using a collimated and precisely controlled X-ray beam
to scan photodetectors in an optically opaque cryostat subject to
mechanical stress and motion.  The photodetectors in the central axial
region $|Z|<120$~mm on the entrace face of the calorimeter are scanned
with average precision of 0.17~mm (0.3~mm for the region $|Z|<150$~mm)
in the axial (Z) and azimuthal ($\phi$) coordinates, outperforming the
resolution goals of the upgraded experiment.  The position map
produced as a result shows systematic displacement and, installation
artifacts observed in the form of rotation and deformation of the CFRP
assembly boards, compared to the nominal grid pattern of the
photodetectors.  Photodetectors in the outer axial region
$|Z|>150$~mm, approximately 54\% of the total, were excluded due to
high X-ray absorption by the magnet.  The calculated absolute
positions serve as a critical input for photon reconstruction in
physics analysis.  This study presents a successful implementation of
the X-ray survey technique in the MEG experiment. It demonstrates a
model for performing systematic position survey of photodetectors in
other scenarios where optical survey methods are not viable or
applicable.

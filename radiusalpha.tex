\subsection {R coordinate calculation and 
coefficient of thermal expansion}
The radial coordinate not directly measured in the X-ray scan is
calculated using information from the FARO scan of the photodetectors.
The FARO scan provides highly precise 3D coordinates
($\sigma_{|\vec{x}|}$ = 200 \micron) of a subset of photodetectors
(10\%) measured at room temperature. To determine the position of
every photodetector, these measurements are interpolated by fitting
the 3D surface formed by the photodetectors. The radial coordinate and
center are calculted first with a cylindrical fit function, followed
by z,$\phi$ coordinates fitted to a regularly spaced grid in z-$\phi$
plane allowing for small rotations in the plane.  Each cfrp plate is
fitted independently since the curvature varies slightly between
individual plates and the overall curvature of the calorimeter.  


The interpolated photodetector locations from FARO
and X-ray scans are fitted with degrees of freedom to account for
global translational motion, extrinsic rotation centered in the MEG
coordinate  system, and scale factor for thermal contraction. 
The FARO coordinate is transformed using the equation
\begin{align}
\vec{x}_{FARO}^{'} = (1-s)R(\alpha,\beta,\gamma)\vec{x}_{FARO} +
\vec{x}_{offset},  
\end{align}
where $s$ is the scale of thermal contraction, $R$ is the rotation
matrix  and $\vec{x}_{offset}$ is the linear offset. 
The fit parameter values are extracted by minimizing the following function
\begin {align}
\chi^2 = 
\sum\limits_{i}^{N_{MPPC}} \frac{(Z_{i}^{Xray}-Z_{i,FARO}^{'})^2}{\sigma_{Z_{i}}^2} 
+ 
\frac{(\phi_{i}^{Xray}-\phi_{i,FARO}^{'})^2}{\sigma_{\phi_i}^2},
\end{align}
where $\sigma_{Z_i}$ and $\sigma_{\phi_i}$ include the resolution
of X-ray and FARO measurements
added in quadrature: 
$\sigma^{FARO}$ = 0.14~mm (0.25~mrad), 
$\sigma^{Xray}$ = 0.40~mm (0.56~mrad),
calculated from adjacent MPPC spacing described above.


The radial values from the FARO measurement are transformed using
fitted parameters to calculate radial coordinate of each
photodetector.  The uncertainty is calculated using linear error
propagation and corresponds to change in the main result due to one
standard deviation variation in the parameters taking correlations
into account.  The results in table \ref{tab:radius} shown mean radius
648.7 and 644.9~mm,and mean error 0.16 and 0.27~mm respectively in the
two scans(figure~\ref{fig:radiuscalculation}).  The radial coordinate
change of 3.8~mm between the two scans is consistent with the shift of
equal magnitude (3.85$\pm$0.1~mm) in the X-coordinate of the LXe
calorimeter in the same period.  A $\phi$ dependent change in the
radial coordinate is seen in 2018 data compared to the previous year's
data due to the calorimeter center offset with respect to the X-ray
device in 2018.



The thermal expansion coefficient is calculated as
\begin{align}
 s= \alpha_t \, \Delta T,
\end{align}
where s is the scale of thermal contraction, $\alpha_t$ is coefficient
of thermal expansion and $\Delta T$ is the change in temperature. 
Using $\Delta T$ ( = 123$\pm$10 K) 
between FARO survey performed at room temperature
(293 K) and X-ray survey performed at LXe temperature (170 K)
the thermal coefficient is calculated in the
two X-ray surveys (table \ref{tab:thermalcoefficient}).
\begin{table}[h]
\centering
\begin{tabular}{ccc}
Scan Period & $s$ & $\alpha_t \,\, [\mathrm{ppm K}^{-1}]$\\
\hline
2017 & 0.0014(1) & 11.0$\pm$1.4 \\
2018 & 0.0018(2) & 14.5$\pm$2.2 
\end{tabular}
\caption{Fitted scaling parameter ($s$) and the calculated thermal 
expansion coefficient.}
\label{tab:thermalcoefficient}
\end{table}
The theoretical value of $\alpha_t$ for the
detectaor material is 16$\pm$1 ppmK$^{-1}$.



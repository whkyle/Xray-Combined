\subsection{Spacing in Z and $\phi$}
The photodetectors are manufactured and assembled in lattice formation
to high precision, which can be used for validation of the X-ray
scanning technique and calculate its resolution.  A change in the
spacing can also reveal the effect of thermal cooling as well as
installation errors.  The spacing between individual MPPC pairs as
well as mean spacing between adjacent MPPC is calculated per PCB strip
(row-wise) along Z and per cfrp plate (column-wise) along \phis. 

Table \ref{tab:oddeven} shows spacing between individual MPPC pairs separated
into two groups which include alternate pairs along columns for Z measurement,
and rows for $\phi$ measurement.  Nominally we expect the spacing to be equal in
the two groups; the data shows the spacing in Z coordinate differs by 0.88~mm
and 0.74~mrad in the $\phi$ coordinate.  The average spacing over the scanned
region (shown in table \ref{tab:avgspacing}), however, is consistent with
expectation after accounting for thermal cooling.  
This implies a systematic shift or miscalculated position 
of one set of every other MPPC by 0.22~mm (0.18 mrad), and the oppposite 
shift for the complementary set, such there is not net displacement over the entire
row or column. 

The source of this discrepancy is not known.
Two possible causes were investigasted that could introduce observed 
systematic shifts in reconstructed MPPC positions with the periodicty of 
$\sim$30 mm, or two MPPC widths: X-ray corrections and trigger bias.
Although inaccurate corrections to the X-ray position can introduce the magnitude of 
displacement in MPPC position observed the periodicity of the corrections
does not match that of the observed effect; Z corrections have a period of 
100~mm while $\phi$ corrections are monotonic.
Similarly, studies of the effect of trigger configuration on the reconstructed position 
were not able to replicate this effect. 

As the scale of the systematic shift is small compared to the overall
X-ray measurement uncertainty as well as the precision goals of the
MEG experiment, it is expected to have minimal effect on the photon
position resolution.  We estimate this effect on photon reconstrution
using MC simulation.  A photon interaction is simulated at
normal distance equivalent to one radiation length
($\lambda_{LXe}\approx$ 2.75 cm) over a two dimensional surface
consisting of (20 $\times$ 20) photodetector array.  The dimensions and
spacing of the photodetectors are identical to the MPPC 
in the LXe calorimeter.  The isotropic distribution of photons generated as a
result is recorded by the photodetector array. The signal in each
photodetector is proportional to the solid angle subtended to the
interaction point, and detection efficiency determined by the photon
incident angle, measured in standalone tests.  
Each event is simulated with an interaction point
chosen randomly in a 30$\times$30~mm$^2$ region equivalent to two MPPC
widths, and reconstructed twice, with nominal and systematically
shifted photodetector positions by 0.22~mm.  The photon position is
reconstructed using weighted mean of the signal recorded in the
photodetectors.
The results show no mean displacement, while the photo 
position resolution is degraded by 0.08~mm in the 
measurement plane.

The mean distance between adjacent MPPCs is calculated using only the
measurements at the end of the row or column.  The precision of the
measurement is improved as a result compared to the pair-wise spacing
by  one (two) order of magnitude in Z ($\phi$).  In the Z coordinate,
the mean distance is calculated for each PCB strip as well as for the
half-strips (US and DS) connected at the center.  Similarly in the
$\phi$ coordinate, the mean distance is calculated within individual
cfrp plates and overall.  The mean and standard deviation of the
calculated mean distance in Z and $\phi$  show a regular grid
structure with no deformation in the scanned region (table
\ref{tab:avgspacing}).

The effect of thermal cooling is seen in Z with the mean distance
15.08~mm  or 30~\micron contraction between adjacent MPPCs. The mean
distance in $\phi$ is dependent on the X-coordinate of the
semi-cylindrical LXe calorimeter whose center is nominally aligned
with the center of the X-ray scanning device.  During the X-ray survey
the calorimeter was off-center, shifted towards the X-ray device by
3.85~mm increasing the mean angular spacing seen in the data.  The
following sections detail the qualitative use of X-ray data to measure
thermal contraction and 3D MPPC location.

\begin{table}
\centering
\begin{tabular}{ccccc}
 & $Z_{n}-Z_{n-1}$ &Std. Dev.& $\phi_{n}-\phi_{n-1}$ & Std. Dev. \\
 & [mm] &[mm]& [mrad]& [mrad]\\
\hline
Odd  $n$ & 15.16 & 0.32 & 23.71 & 0.78 \\ 
Even $n$ & 14.98 & 0.35 & 23.69 & 0.57 \\ 
%Nominal  & 15.1  &      & 23.55 & \\
\end{tabular}
\caption{Mean Z and $\phi$ spacing calculated for Odd-Even and Even-Odd
combination  of adjacent MPPCs.}
\label{tab:oddeven}
\end{table}

\begin{table}
\centering
\begin{tabular}{clc}
   & Z [mm] &Std. Dev. [mm] \\
\hline
US     & 15.10 & 0.05  \\
DS     & 15.06 & 0.03  \\
All    & 15.06 & 0.02  \\
Nominal& 15.1      &
\end{tabular}

\begin{tabular}{clc}
   & $\phi$ [mrad] &Std. Dev. [mrad] \\
\hline
cfrp 1     & 23.72 & 0.08  \\
cfrp 2     & 23.70 & 0.01  \\
cfrp 3     & 23.71 & 0.04  \\
cfrp 4     & 23.73 & 0.04  \\
All        & 23.72 & 0.01  \\
Nominal    & 23.55 &     
\end{tabular}
\label{tab:avgspacing}
\caption{Mean distance between adjacent MPPC calculated over the entire row (Z) or 
column ($\phi$). Upstream (US) and downstream (DS) parts of each PCB strip, and
individual cfrp plates are calculated separately, and altogether.}
\end{table}



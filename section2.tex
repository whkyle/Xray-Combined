\section{\label{motivation}Motivation}
Relative alignment of the photo-detectors and the positron
spectrometer is essential in MEG to realize the acceptance
requirements of \mueg topology.  Although an alignment
using external markers on the LXe calorimeter is performed the
relative positions of the markers and the photo-detectors is
subject to change. The photo-detectors are enclosed in a
cryostat and the volume of the inner 
calorimeter is filled with liquid Xenon and cooled to 170~K. 
The in-situ position of the
detectors is affected by thermal contraction and buoyant
forces which cannot be easily modeled or predicted. In the
previous MEG run a linear deformation model was used to
correct for these effects which is no longer applicable due
to stringent resolution goals\cite{megdesign,megproposal}
(u$_\gamma$/v$_\gamma$ 5/5~mm (Run I) vs 2.6/2.2 mm (Run II))
of this run for photon reconstruction.  An increased photosensitive
area in the upgraded calorimeter requires a systematic survey
that can scan with higher precision than possible with the
studies using charge exchange reaction \cex in an stationary
liquid hydrogen target, conducted in Run~I.  An X-ray survey of
the detector is conceived to determine the absolute position
of the photo-detectors inside the LXe calorimeter. The design,
operation and data analysis with the X-ray technique is
documented in the following sections.




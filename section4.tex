\section{\label{sec:technique}Overview of technique and apparatus}
\subsection{Outline of technique}
The X-ray technique measures the axial (Z) and azimuthal ($\phi$)
position of each photodetector (or MPPC) by scanning individual
photodetector with a well collimated beam of X-ray. The beam is
precisely located in the azimuthal and axial coordinates, and pointed
radially originating from the central axis of MEG (Figure
\ref{fig:photo}).  The collimator projects a thin (1$\times$30~mm, 1.5$\times$50
mrad$^2$) beam on the inner face of cryostat, about 8\% the width of
the MPPC in the narrow dimension, whose motion is controlled by
precise linear and rotation stages. A coordinate of the
MPPC is calculated by orienting the narrow dimension of the beam in
that direction and scanning. The signal from the scintillating light
as a function of X-ray beam position determines the spatial extent of
the MPPC. The spatial resolution as well as exposure time per position
is chosen to balance the desired statistical precision and the time
budget allocated for the calorimeter scan. A test X-ray scan of cubic
(6~mm) LYSO scintillator is shown in figure \ref{fig:beamprofile}.

The X-rays are produced by the decays of \cob source which produces
two X-ray emission lines at 122 keV ($\approx$80\%) and 136 keV
($\approx$10\%).  A commercially produced point source with
activity of 3$\times 10^{10}$ Bq is used. The energy of source is
chosen such that it penetrates the magnetic and calorimetric cryostats
but low enough to interact with liquid Xe within $\approx$3~mm. The light
detected by the MPPCs is produced close to the surface of the
photodetector primarily by photo-absorption, and generates signal of
about 30\% that of a typical 53 MeV photon from \mueg events. Raw
waveform and pulse-height distributions for a given channel
illuminated with and without X-rays show the magnitude of the observed
signal (Figure \ref{fig:pulseheight}).

The precision of the measurement depends on the precision with which the X-ray
coordinates can be determined.  The apparatus is aligned by means of optical
survey so that the X-ray motion is coaxial with the COBRA magnet centered in
the MEG coordinate system, and small variations in the X-ray position are
monitored with a laser system and a bubble-level.  We rely on the position
information from the motion stages which is corrected for known angular and
linear misalignments.  An additional cross-check of the X-ray position
calculations is performed by placing precisely aligned lead absorber strips
(30$\times$3~mm$^2$) on the front face of LXe cryostat.  The lead positions are
determined with optical survey and independently with supressed X-ray events
(shadow) in the scanning of the photodetectors.  The agreement between these
measurements is used for validaiton of X-ray alignment and correction
procedures, and serves as the upper limit of the precision of the measurement. 

A high precision 3D coordinate meaurement of the MPPCs is carried out with a
FARO machine \cite{faro} in an open, unfilled calorimeter at room temperature.
This serves as a complimentary measurement to the X-ray survey, and a benchmark
for studying the effects of cryogenically cooled calorimeter on the
photodetectors.
\begin{figure}[]
\includegraphics[width=8cm]{plots/photo4}
\includegraphics[width=8cm]{plots/photo3}
\caption{X-ray survey apparatus shown mounted inside COBRA magnet
  (left) and outside the experimental area (right).}
\label{fig:photo}
\end{figure}  

\begin{figure}[]
\includegraphics[width=4cm]{plots/xray_lyso.pdf}
\includegraphics[width=4cm]{plots/xray_lyso3_jun5.pdf}
\caption{Beam profiles in the LYSO scintillator placed on the outer
  wall of COBRA magnet.}
\label{fig:beamprofile}
\end{figure}  

\begin{figure}[]
\centering
\includegraphics[width=4cm]{plots/2018/XrayWF_MPPC67}
\includegraphics[width=4cm]{plots/2018/Sel2_2}
\caption{Recorded waveform amplitude in a single channel for X-ray and background events.}
\label{fig:pulseheight}
\end{figure}  


\subsection{X-ray Beam Production}
The X-ray system consists of a collimator placed on linear and rotary stages to
provide motion in Z and \phis direction.  A system of optical laser and
quadrant photodiode, and another of camera and bubble-level are used to
continuously monitor the motion of the collimator  with sub-millimeter
precision.  A Raspberry Pi device performs i/o operation with the motion
controllers of the motion stages, and transfers the readout data from the
controllers, the QPD and the camera to the MEG DAQ system \cite{megsoftware}.
A software patch was written to enable the existing MIDAS software framework
\cite{midas} used for MEG DAQ operations to communicate with the Raspberry Pi
over Gigabit Ethernet.  Downstream processing and analysis is handled by the
DAQ system.
 


